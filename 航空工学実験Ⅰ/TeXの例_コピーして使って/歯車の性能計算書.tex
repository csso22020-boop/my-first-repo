\documentclass[a4paper,titlepage]{ltjsarticle}
\usepackage{luatexja} 

\usepackage{amsmath, amssymb, ar, bm, booktabs, multirow, siunitx, url, graphicx
}

\newcommand{\hang}[1]{%
	\settowidth{\hangindent}{#1}%
	#1%
}%


\newenvironment{hangall}[1]{\hangindent = #1zw\everypar{\hangindent = #1zw}}{}

\graphicspath{{picture}}

\begin{document}

\title{設計製図Ⅰ \\第11回課題\\歯車の性能および強度計算書}
\author{2年 2020番 大村蒼摩}
\date{\today}
\maketitle

\setlength{\abovedisplayskip}{15pt} 
\setlength{\belowdisplayskip}{15pt}


\vskip\baselineskip

\begin{table}[hbtp]
	\centering
	\caption{性能および強度計算結果}
	\label{}
	\begin{tabular}{l|c}
		\toprule
		中間軸回転数$n_{2}{\si{[rpm]}}$&381\\
		出力軸回転数$n_{3}{ \si{[rpm]}}$&76.1\\
		1段目減速比$i_{1}{ \si{[-]}}$&0.254\\
		2段目減速比$i_{2}{ \si{[-]}}$&0.200\\
		全減速比$i{ \si{[-]}}$&0.0507\\
		入力軸の動力$N${ \si{[W]}}&283\\
		入力軸トルク$T_{1}{ \si{[Nm]}}$&1.80\\
		中間軸の動力$N_{2}{ \si{[W]}}$&268\\
		中間軸のトルク$T_{2}{ \si{[Nm]}}$&6.73\\
		出力軸の動力$N{ \si{[W]}}$&255\\
		出力軸のトルク$T_3{ \si{[Nm]}}$&32.0\\
		入力軸の基準ピッチ円直径$D_{1}{ \si{[mm]}}$&22.0\\
		中間軸1の基準ピッチ円直径$D_2{ \si{[mm]}}$&86.7\\
		中間軸2の基準ピッチ円直径$D_3{ \si{[mm]}}$&18.1\\
		出力軸の基準ピッチ円直径$D_4{ \si{[mm]}}$&90.6\\
		軸間距離$a{ \si{[mm]}}$&54.4\\
		ピッチ円接線方向の力$F{ \si{[N]}}$&706\\
		最大曲げ応力$\sigma_{bmax}{ \si{[MPa]}}$&127\\ 
		安全率&1.74\\
	 \bottomrule
	\end{tabular}		
\end{table}

\begin{table}[hbtp]
	\centering
	\caption{減速機の仕様}
	\label{}
	\begin{tabular}{l|c}
		\toprule
		入力軸回転数$n_1$&1500 \si{rpm}\\
		伝達効率$\eta $& 90\%\\
		中間軸と出力軸の伝達効率$\eta _2$&95\%\\
		減速比$i$&1/20\\
		出力トルク$T_3$&32.0 \si{Nm}\\
	 \bottomrule
	\end{tabular}
\end{table}

\clearpage
\begin{table}[hbtp]
	\centering
	\caption{歯車の仕様}
	\label{}
	\begin{tabular}{l|cccc}
		\toprule
		&入力ピニオン$a_1$&高速ギア$a_2$&出力ピニオン$b_1$&低速ギア$b_2$\\
		歯数&17&67&14&70\\
		モジュールm&1.25&1.25&1.25&1.25\\
		圧力角$\alpha _0$&20°&20°&20°&20°\\
		ねじれ角\beta &$\ang{15}$&$\ang{15}$&$\ang{15}$&$\ang{15}$\\
	 \bottomrule
	\end{tabular}
\end{table}

\section{中間軸回転数$n_2$[rpm],出力軸の回転数$n_3$[rpm]}

中間軸の回転数の$n_2$の算出を行う.単位時間あたりの噛み合う歯の数は等しい.

\begin{equation}
	n_1z_1=n_2z_2
\end{equation}
と表される.そのため$n_2$は
\begin{equation}
	n_2=n_1\frac{z_{a1}}{z_{a2}}
\end{equation}
となる.
$z_{a1}$=17,$z_{a2}$=67,$n_1$=1500 \si{rpm}であるから
\begin{align}
	n_2&=1500{ \si{rpm}} \times\frac{17}{67}\\
	   &=380.597\\
		 &\fallingdotseq381{ \si{rpm}}
\end{align}

次に出力軸の回転数を求める.$n_2$のときと同様に
\begin{equation}
	n_3=n_2\frac{z_{b1}}{z_{b2}}
\end{equation}
となり,
$n_2$=380.6,$z_{b1}$=14,$z_b2$=70より
\begin{align}
	n_3&=380.6\times\frac{14}{70}\\
	   &=76.12 \\
		 &=76.1{ \si{rpm}}
\end{align}

\section{1段目減速比$i_1$,2段目減速比$i_2$,全減速比$i$}

\subsection{1段目の減速比}
まず1段目の減速比を求める.減速比(回転数の比)は
\begin{equation}
	i=\frac{n_2}{n_1}=\frac{D_1}{D_2}=\frac{z_1}{z_2}
\end{equation}
と表せる.よって,回転数$n_1$,$n_2$を用いて減速比を計算すると
式(9)より,$n_1$=1500,$n_2$=380.6なため
\begin{align}
	i_1&=\frac{380.6}{1500}\\
	   &=0.25373\\
		 &\fallingdotseq0.254
\end{align}
となる.

\subsection{2段目の減速比}
同様に$i_2$は$i_2=\frac{n_3}{n_2}$であり,$n_2$は式(5),$n_3$は式(8)より
\begin{align}
	i_2&=\frac{76.12}{380.6}\\
	   &=0.200
\end{align}
となる.

\subsection{全減速比}
全減速比は$i=\frac{n_3}{n_1}$,$n_3$は式(8)より
\begin{align}
	i&=\frac{76.12}{1500}\\
	 &=0.050746\\
	 &\fallingdotseq0.0507
\end{align}

\section{入力軸,中間軸,出力軸の動力$N${\si{[W]}}とトルク$T${\si{[Nm]}}}

\subsection{入力軸の動力}
回転する動力は,
{動力=トルク$\times$回転角速度}
で算出できる.
\begin{equation}
\label{}
N=T_3\times(n_3\frac{2\pi}{60})=\frac{\pi n_3T_3}{30} \si{W}
\end{equation}
で全体の動力が求められる.$T_3$=32.0{Nm},$n_3$=76.1よりNを算出すると
\begin{align}
\label{a}
N &=32.0{\si{Nm}}\times (76.1\times \frac{2\pi}{30})\\
  &=254.6784\\
	&\fallingdotseq 255{\si{W}}\\
\end{align}

そこから、各軸の動力Nを求める.
最初に入力軸動力$N_1$は
\begin{equation}
	N_1=\frac{N}{\eta _1}
\end{equation}
で表すことができる.(N:全体の動力,$\eta $:伝達効率)式(19)を用いて入力軸の動力を求める.
\begin{align}
	N_1&=\frac{255}{0.9}\\
	&=283.33\\
	&\fallingdotseq 283 \si{W}
\end{align}

中間軸動力は
\begin{align}
\label{a}
N_2 &=\frac{N}{\eta _2}\\
&=\frac{255}{0.95}\\
&=268.421\\
&\fallingdotseq 268 {\si{W}}
\end{align}

次にトルクを求める.
入力軸トルクは
\begin{align}
\label{}
T_1=\frac{30N_1}{\pi n_1}
\end{align}
で求められる.
よって、入力軸トルクは
\begin{align}
\label{}
T_1&=\frac{30\times 283.33}{\pi \times 1500}\\
&=1.8035\\
&\fallingdotseq 1.80{ \si{Nm}}
\end{align}

次に、中間軸トルクは
\begin{align}
\label{}
T_2=\frac{30N_2}{\pi n_2}
\end{align}
で求められ、実際に算出してみると

\begin{align}
\label{}
T_2&=\frac{30 \times268.42}{\pi \times 381}\\
&=6.7276\\
&\fallingdotseq 6.73{\si{Nm}}
\end{align}

出力軸トルクは,表2より$T_3=32.0{ \si{Nm}}$

\section{入力軸,中間軸,出力軸の基準ピッチ円直径$d$[mm]}
基準ピッチ円直径は以下の式で求められる.
\begin{align}
\label{}
d=\frac{zm}{\cos\beta }
\end{align}
で表される.

\subsection{入力軸の基準ピッチ円直径}
式(37)より
\begin{align}
	D_1&=\frac{17{ \si{mm}}\times1.25}{\cos\ang{15}}\\
	   &=21.996 {\si{mm}}\\
		 &=22.0 {\si{mm}}
\end{align}

\subsection{中間軸の歯車1基準ピッチ円直径}
式(37)より
\begin{align}
	D_2&=\frac{67 {\si{mm}}\times1.25}{\cos\ang{15}}\\
	   &=86.7{ \si{mm}}\\
\end{align}

\subsection{中間軸の歯車2の基準ピッチ円直径}
式(37)より
\begin{align}
\label{}
D_3&=\frac{14{\si{mm}}\times 1.25}{\cos\ang{15}}\\
&=18.1173{\si{mm}}\\
&=18.1{\si{mm}}
\end{align}

\subsection{出力軸の歯車の基準ピッチ円直径}
\begin{align}
\label{}
D_4&=\frac{70{\si{mm}}\times1.25}{\cos\ang{15}}\\
   &=90.586{ \si{mm}}\\
   &\fallingdotseq 90.6 {\si{mm}}
\end{align}

\section{軸間距離}
はすば歯車の軸間距離は
\begin{equation}
\label{}
a=\frac{m(z_1+z_2)}{2\cos\beta }
\end{equation}
で表される.m:モジュール,$z_1,z_2$:歯数
式(49)より,入力軸と中間軸の軸間距離は
\begin{align}
\label{}
a&=\frac{1.25 {\si{mm}}\times(17+67)}{2\times \cos\ang{15}}\\
 &=54.352\\
 &=54.4 {\si{mm}}
\end{align}

次に,中間軸と出力軸の軸間距離を算出する.
\begin{align}
\label{}
a&=\frac{1.25{\si{mm}}\times(14+70)}{2\times \cos\ang{15}}\\
&=54.352\\
 &=54.4 {\si{mm}}
\end{align}

以上で軸間距離が求められた.

\section{強度計算}
\begin{table}[hbtp]
	\centering
	\caption{文字の定義}
	\label{}
	\begin{tabular}{l|c}
		\toprule
		文字&数値\\
		\hline
		T[Nm]&6.7368\\
		D[mm]&90.6\\
		b[mm]&10\\
		m[mm]&1.25\\
		Y[-]&2.2\\
	 \bottomrule
	\end{tabular}
\end{table}
まず,歯車$b_2$ピッチ円接線方向の力を求める.
式は以下のように表すことができる.
\begin{equation}
\label{}
F=\frac{T}{\frac{D}{2}}<{\text 許容される値}
\end{equation}
この式を用いて$F$を算出する.
\begin{align}
\label{}
F&=\frac{32.0}{\frac{90.6\times10^{-3}}{2}}\\
&=706.401\\
&\fallingdotseq 706\si{N}
\end{align}

最大曲げ応力$\sigma _{b max}$は以下の式で求められる.
\begin{align}
\label{}
\sigma_{b max}=\frac{FY}{bm}
\end{align}
この式を用いると
\begin{align}
\label{}
\sigma_{bmax}&=\frac{706.401\times 2.22}{10 \times 1.25}\\
&=127.152\\
&=127{\si{MPa}}
\end{align}

安全率は,\\

\begin{equation}
\label{}
\text{安全率}=\frac{\text 曲げ疲労限度}{\text{最大曲げ応力}}
\end{equation}

より求められる.P.139の表6.7より曲げ疲労限度は$221 \si{Mpa}$
よって,
\begin{align}
\label{}
\frac{221}{127}&=1.740\\
&=1.74\\
&\fallingdotseq 1.74
\end{align}
このような結果から,安全率1.5を満たす.


\end{document}

